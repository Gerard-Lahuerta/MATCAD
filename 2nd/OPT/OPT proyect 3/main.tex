% Nom Cognom
\documentclass[a4paper, 11pt]{article}
\usepackage[utf8]{inputenc} %alternativa: [latin1]
\usepackage[T1]{fontenc}
\usepackage[catalan]{babel} %alternativa: altres idiomes
\usepackage{amsmath, amssymb, amsthm}
\usepackage[margin=1in]{geometry}
\usepackage{enumerate}
\usepackage{array}
\usepackage{graphicx}
\usepackage{wrapfig}
\usepackage{ragged2e} 
\usepackage{subfig}
\usepackage{caption}
\usepackage{subcaption}
%\usepackage[usenames]{color}
\usepackage[dvipsnames]{xcolor}
\usepackage{float}
\usepackage{chngcntr}
\usepackage{ragged2e}
\usepackage{multirow}
\usepackage{vmargin}
\usepackage{hyperref}
\usepackage{url}
\usepackage{fancyhdr}
\usepackage{bigints}

\setpapersize{A4}
\setmargins{2.5cm}     % margen izquierdo
{2.6cm}                % margen superior
{16.5cm}               % anchura del texto
{23.7cm}               % altura del texto
{10pt}                 % altura de los encabezados
{0cm}                  % espacio entre el texto y los encabezados
{0pt}                  % altura del pie de página
{1cm}                  % espacio entre el texto y el pie de página
\renewcommand{\baselinestretch}{1.5}


\begin{document}

\begin{titlepage}
    \centering
    {\bfseries\LARGE Universitat Autònoma de Barcelona \par Facultat de Ciències\par}
    \vspace{2cm}
    \vspace{1cm}
    {\scshape\Huge Lliurament Pràctica 3 d'Optimització \par} 
    \vspace{1cm}
    {\Large \itshape Autor: \par}
    {\large Gerard Lahuerta 1601350\par}
    \vspace{1cm}
    {\Large 25 de Març del 2022\par}
\end{titlepage}

\justifying
\newpage
\section{Comparativa del Mètodes}
En aquesta pràctica hem hagut de programar 5 diferents tipus de mètodes d'Optimització lineal que s'inclouen a la libreria \textit{GSL}. \\
Aquest mètodes han sigut Nelder-Mead, Steepest descent, BFGS, Fletcher-Reeves i Polak-Ribiere.\\
Hem estudiat el seu comportament amb un programa en llenguatge \texttt{C} que té com a objectiu trobar el mínim de la funció $f(x): \mathbb{R}^n \longrightarrow \mathbb{R}$ amb $n=10$, punt inicial $x_0=(1,1,1,1,1,1,1,1,1,1)$ i
\begin{equation*}
    f(x) = \sum_{i=2}^{n}\left[(1+p)\cdot (x_{i-1}-3)^2 + (x_{i-1}-x_i)^2 + e^{20(x_{i-1}-x_i)}\right] \text{  on $p=1,1$.}
\end{equation*}
Els resultats obtinguts han sigut els següents:
\begin{center}
  \begin{tabular}{ c | c | c }
    \textbf{Mètode} & \textbf{Iteracions} & \textbf{Convergencia} \\ \hline
     Nelder-Mead & 2350 & (2.61, 2.73, 2.83, 2.91, 2.99, 3.08, 3.16, 3.26, 3.38, 3.58) \\ \hline
     Steepest descent &  8 & (1.00, 1.00, 1.11, 1.17, 1.15, 1.21, 1.21, 1.24, 1.27, 1.29) \\ \hline
     BFGS & 28 & (2.47, 2.58, 2.62, 2.76, 2.95, 3.42, 3.60, 3.69, 3.88, 5.09)  \\ \hline
     Fletcher-Reeves & 46 & (2.94, 3.01, 3.05, 3.11, 3.14, 3.42, 3.52, 3.63, 3.79, 5.21) \\ \hline
     Polak-Ribiere & 28 & (2.47, 2.58, 2.62, 2.76, 2.95, 3.42, 3.60, 3.69, 3.88, 5.09) \\ \hline
  \end{tabular}
\end{center}

\section{Anàlisis del resultats i Conclusions}
Observem que el punt de convergencia entre els mètodes no coincideix pel que experimentem una altre vegada pero cambiant el punt incial per $x_0 = (2.47, 2.58, 2.62, 2.76, 2.95, 3.42, 3.60, 3.69, 3.88, 5.09)$ per comprovar si el mètode funciona correctament i el que succeeix es que els mètodes tendeixen a mínims locals diferents degut a les diferéncies entre els algoritmes.
\begin{center}
  \begin{tabular}{ c | c | c }
    \textbf{Mètode} & \textbf{Iteracions} & \textbf{Convergencia} \\ \hline
     Nelder-Mead & 616 & (2.62, 2.73, 2.83, 2.91, 2.99, 3.08, 3.16, 3.26, 3.38, 3.57) \\ \hline
     Steepest descent & 6 & (2.80, 2.93, 3.06, 3.13, 3.25, 3.30, 3.42, 3.57, 3.78, 5.08) \\ \hline
     BFGS & 30 & (2.34, 2.65, 2.76, 2.85, 2.94, 2.99, 3.11, 3.24, 3.44, 3.95) \\ \hline
     Fletcher-Reeves & 56 & (2.63, 2.74, 2.83, 2.91, 3.00, 3.07, 3.16, 3.26, 3.37, 3.63) \\ \hline
     Polak-Ribiere & 28 & (2.47, 2.58, 2.62, 2.76, 2.95, 3.42, 3.60, 3.69, 3.88, 5.09) \\ \hline
  \end{tabular}
\end{center}
Demostrem per tant la nostra hipòtesis i que les diferències son degudes a les variacions de la propia funció i els diversos mínims locals que té.\\
Per altra banda ens percatem de les diferencies en la quantitat d'iteracions necesaries per a trobar un mínim; de millor a pitjor és: \textit{Steepest Descend}, \textit{Polak-Ribiere}, \textit{BFGS}, \textit{Fletcher-Reeves} i eventualment \textit{Nelder-Mead}.
\newpage
\textcolor{white}{q}
\end{document}